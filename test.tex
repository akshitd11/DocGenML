
\documentclass{article}
\usepackage{hyperref}
\usepackage{listings}
\usepackage{xcolor}
\usepackage{geometry}

\geometry{margin=1in}

\title{Code Documentation}
\author{Generated by DocGen}
\date{\today}

\begin{document}

\maketitle
\tableofcontents


\section{cli}








\section{cli}


docgen: generate docs & summarize code







\section{generate}


Generate docs from comments & docstrings







\section{summarize}


Summarize a .py or .ipynb file using OpenAI







\section{full}


Generate docs for all .py files under SRC, summarize every .py and .ipynb there,
and write a single file to OUT in the specified format.







\section{config}








\section{formatter}








\section{BaseFormatter}


Base class for all formatters







\section{render}


Convert parsed items into the target format







\section{get_template}


Load a Jinja2 template from the templates directory







\section{MarkdownFormatter}


Format parsed items as Markdown







\section{render}








\section{HtmlFormatter}


Format parsed items as HTML







\section{render}








\section{LaTeXFormatter}


Format parsed items as LaTeX







\section{render}








\section{PDFFormatter}


Format parsed items as PDF (via ReportLab)







\section{render}








\section{get_formatter}


Factory function to get the appropriate formatter







\section{generator}








\section{MarkdownGenerator}








\section{render}








\section{parser}








\section{DocItem}








\section{extract_comments}


Map ending line number → collected comment block above it.







\section{parse_file}








\section{summarize}








\section{read_code_from_file}


Read a Python (.py) or Jupyter notebook (.ipynb) file and return its code as a single string.
- .py: returns the entire file.
- .ipynb: concatenates all code-cell sources, separated by blank lines.







\section{split_code_by_function_or_class}


Split the full code into top‑level chunks, each beginning with a `def ` or `class `.
Returns up to `max_chunks` segments to avoid overly long prompts.







\section{build_combined_prompt}


Build a single user prompt for the LLM by enumerating each chunk.
Prepends a header comment "# Chunk i" to each segment for clarity.







\section{summarize_code_file}


Orchestrates the full summarization:
  1. Reads code from the given file.
  2. Splits into manageable chunks.
  3. Builds a combined prompt.
  4. Calls the OpenAI API to get a summary.
  5. Returns the cleaned summary text.







\section{__init__}








\section{Summary of cli.py}


The codebase provides a command-line interface (CLI) tool for generating documentation from Python files' comments and docstrings. It also includes functionality to summarize Python or Jupyter notebook files using OpenAI and can output the combined documentation and summaries in various formats like markdown, HTML, PDF, or LaTeX.







\section{Summary of config.py}


This code chunk loads environment variables from a .env file and sets default values for variables related to OpenAI API configuration.







\section{Summary of formatter.py}


This codebase defines a system for formatting parsed items into various document formats (Markdown, HTML, LaTeX, PDF) using different subclasses of a base formatter class. It uses Jinja2 for templates and ReportLab for PDF generation. A factory function is provided to get the appropriate formatter based on the desired output format.







\section{Summary of generator.py}


The code consists of a MarkdownGenerator class that generates markdown documentation for a list of DocItem objects. The render method creates markdown content for each DocItem based on its type, name, comments, and docstring.







\section{Summary of parser.py}


These code chunks define classes and functions to parse Python code files, extract documentation items (module, class, or function) along with their docstrings and comments, and organize them into a structured format for further processing. The code includes functions to extract comments, parse a file, and generate DocItem objects containing relevant information from the parsed code.







\section{Summary of summarize.py}


The code provided is a Python script that reads code from a file (either a Python script or a Jupyter notebook), splits the code into manageable chunks based on functions or classes, builds a combined prompt for an AI model, and then uses the OpenAI API to generate a summary of the code. The main function, `summarize_code_file`, orchestrates this process by handling file reading, chunking, prompt creation, API querying, and returning the cleaned summary text. The goal is to automate the summarization of codebases by leveraging OpenAI's language model.







\section{Summary of __init__.py}


I'm sorry, but you have not provided any code chunks for me to summarize. Please provide the code chunks that you would like me to summarize.








\end{document}